\documentclass[chapter3.tex]{subfiles}
\usepackage[overlap, CJK]{ruby}

\begin{document}
\begin{CJK}{UTF8}{ipxm}
    % p.142
    「どう考えたって\ruby{悪戯}{いたずら}だろ。
    真面目に相手をするなんて馬鹿馬鹿しいじゃないか」

    しかし\ruby{老}{お}いた父親は\ruby{一向}{いっこう}に\ruby{懲}{こ}りている様子がなかった。
    それどころか、「お前は何もわかってないなあ」と\ruby{哀}{あわ}れむようにいうのだった。

    何がわかってないのか、とむきになって\ruby{詰問}{きつもん}すると、雄治は\ruby{涼}{すず}しい顔をしてこういった。

    「\ruby{嫌}{いや}がらせだろうが\ruby{悪戯}{いたずら}目的だろうが、『ナミヤ雑貨店』に手紙を入れる人間は、普通の悩み相談者と根本的には同じだ。
    心にどっか\ruby{穴}{あな}が開いていて、そこから大事なものが流れ出しとるんだ。
    その証拠に、そんな連中でも必ず回答を受け取りに来る。
    牛乳箱の中を覗きに来る。
    自分が書いた手紙に、ナミヤの\ruby{爺}{じい}さんがどんな回答を\ruby{寄越}{よこ}すか、知りたくて仕方がないわけだ。
    考えてみな。
    例え\ruby{出鱈目}{でたらめ}な相談事でも、三十も考えて書くのは大変なことだ。
    そんなしんどいことをしておいて、何の答えも欲しくないなんてことは絶対にない。
    だからわしは回答を書くんだ。
    一生懸命、考えて書く。
    人の心の声は、決して無視しちゃいかん。」

    実際に雄治は、その同一人の手によるものと思われる三十通の悩み相談の一つ一つに真面目に回答を書き、朝までに牛乳箱に入れた。
    そして確かに店を開ける前の午前八時には、それらの全てが持ち去られていたのだった。
    その後、\ruby{同種}{どうしゅ}の悪戯は起きていない。
    代わりにある夜、『ごめんなさい。ありがとうございました。』
    と一文だけ書かれた紙が\ruby{放}{ほお}り込まれた。
    その筆跡は、三十通の主のものと\ruby{酷似}{こくじ}していた。
    それを\ruby{誇}{ほこ}らしげに息子に見せた時の父親の顔を、貴之は忘れられない。

    % p.143
    多分生き\ruby{甲斐}{がい}ってやつなんだろうと思った。
    約十年前、貴之の母親が心臓病でこの世を去った時には、雄治はすっかり元気を無くしてしまった。
    すでに子供たちは全員家を出ていた。
    一人きりの孤独な生活は、間も無く七十歳になろうという老人から生きる気力を奪い取るには、十分なほど\ruby{辛}{つら}いものだったようだ。

    貴之には二歳上の、\ruby{頼子}{よりこ}という\ruby{姉}{あね}がいる。
    だが彼女は\ruby{夫}{おっと}の両親と\ruby{同居}{どうきょ}しており、とても\ruby{頼}{たよ}るわけにはいかなかった。
    雄治の面倒を見るとすれば、貴之しかいない。
    とはいえ彼も\ruby{世帯}{しょたい}を持ったばかりの頃だった。
    当時は狭い社宅暮らしで、雄治を引き取る余裕などなかった。

    そんな子供たちの実情をわかっていたのだろう。
    雄治は元気をなくしながらも、店を閉めるとは決して言わなかった。
    貴之も、そんな父のやせ我慢に甘えていた。

    ところがある日、姉の頼子から意外な電話がかかってきた。

    「びっくりしたわよ。
    すっかり元気になってるんだもの。
    お母さんが死ぬ前より生き生きしてるかもしれない。
    あれなら\ruby{一}{ひと}安心。
    当分は大丈夫だと思う。
    あなたも一度顔を見に行ってみたら?
    驚くわよ、きっと」

    久しぶりに様子を見に行ったという姉は、声を\ruby{弾}{はず}ませていた。
    さらに彼女は\ruby{興奮}{こうふん}した口ぶりで、
    「どうしてお\ruby{父}{とう}さんがそんなに元気になったかわかる?」
    と訊いてきたのだ。
    貴之がわからないというと、
    「そりゃそうよねえ、わかるわけないと思う。
    私なんか、それを聞いて二度びっくり」
    と続けた後、ようやく事情を話してくれたのだ。
    お\ruby{父}{とう}さんは悩みの相談室まがいのことをしている、と

    % p.144
    その話を聞いた時、貴之は今ひとつぴんとこなかった。
    何だよそれ、と思っただけだ。
    そこで早速、次の休日に実家に帰ってみた。
    そうして目にした\ruby{光景}{こうけい}は、とても信じられないものだった。
    『ナミヤ雑貨店』の前に人だかりができているのだ。
    集まっているのは主に子供たちだが、大人の姿もあった。
    どうやら彼等は店の壁を\ruby{眺}{なが}めているようだった。
    そこには紙がたくさん貼ってあり、それをみて笑っているのだ。

    貴之は近づいていき、子供たちの\ruby{頭越}{あたまご}しに壁を見上げた。
    そこに貼られているのは便箋やレポート用紙だった。
    小さなメモ用紙もある。
    内容を読んでみると、例えば中の一枚には次のようなことが書かれていた。

    『相談です。勉強せず、カンニングとかのインチキもしないで、テストで百点をとりたいです。
    どうすればいいですか。』

    明らかに子供の\ruby{字}{じ}と思われた。
    それに対する回答が、下に貼られている。
    こちらは貴之が見慣れた雄治の字で書かれていた。

    『先生に頼んで、あなたについてのテストを作ってもらってください。
    あなたのことだから、あなたの書いた答えが必ず正解です。』

    何だこれは、と思った。
    悩みの相談というより、とんちではないか。

    他の悩み相談にも目を\ruby{通}{とお}したが、サンタクロースに来て欲しいが\ruby{煙突}{えんとつ}がないのでどうすればいいかとか、
    地球が\ruby{猿}{さる}の\ruby{惑星}{わくせい}みたいになった時には誰から猿の言葉を習えばいいかとか、
    とにかくどれもこれもふざけた内容ばかりだ。
    だがいずれの質問にも、雄治は\ruby{生真面目}{きまじめ}に回答している。
    どうやらそれがうけているらしい。
    そばには投入口の付いた箱が置いてあり、
    『悩みの相談箱 \;
    どんなことでも遠慮なく相談してください \;
    ナミヤ雑貨店』
    と書いた紙が貼ってあった。

    % p,145
    「まあ、一種の遊びだ。
    \ruby{近所}{きんじょ}のガキ\ruby{共}{とも}の\ruby{挑発}{ちょうはつ}に乗って、引っ込みがつかなくなってやり始めたんだが、
    意外と\ruby{好評}{こうひょう}で、あれを読むために遠くから人が来るようになった。
    何が\ruby{功}{こう}を\ruby{奏}{そう}するかわからんな。
    ただ、\ruby{近頃}{ちかごろ}ではガキ共も\ruby{捻}{ひね}った悩みを入れてきやがるもんだから、こっちも頭を使わなきゃいかん。
    結構大変だ。」

    \ruby{苦笑}{にがわらい}いを浮かべながら話す雄治の表情は、生き生きしていた。
    妻を亡くした直後とは明らかに違っていた。
    姉の言葉は嘘ではなかったのだ。

    雄治の新たな生き\ruby{甲斐}{がい}となった悩み相談は、\ruby{当初}{とうしょ}は遊びの\ruby{要素}{ようそ}が強かったが、やがて真剣な悩みが寄せられるようになった。
    そうなると人目につく相談箱ではまずいだろうということで、現在のシャッターの郵便口と牛乳箱を使った方式に変えたそうだ。
    ただし面白い悩みが持ち込まれた場合には、今まで通り、壁に貼り出しているらしい。

    雄治は\ruby{卓袱台}{ちゃぶだい}の前で\ruby{正座}{せいざ}し、\ruby{腕組}{うでぐ}みをしている。
    便箋を広げているが、ペンを取る気配はなかった。
    \ruby{下唇}{したくちびる}を少し突き出し、\ruby{眉間}{みけん}に\ruby{皺}{しわ}を寄せている。
    
    「随分と考え込んでるな」
    貴之はいった。
    「難しい内容なのか」

    

\end{CJK}
\end{document}
