\documentclass[chapter3.tex]{subfiles}
\usepackage[overlap, CJK]{ruby}

\begin{document}
\begin{CJK}{UTF8}{ipxm}
    雄治はゆっくりと\ruby{頷}{うなず}いた。
    
    「女の人からの相談だ。
    この手の問題は一番\ruby{苦手}{にがて}だ。」
    
    \ruby{色恋}{いろこい}\ruby{沙汰}{ざた}だな、と\ruby{解}{かい}した。
    雄治は見合い結婚だが、お互い\ruby{婚礼}{こんれい}の当日まで相手のことをよく知らなかったという話だ。
    そんな時代を過ごしてきた人間に恋愛問題を相談する方が非常識だと貴之は思う。

    「適当に書いとけよ」

    「何言ってるんだ。
    そんなわけにいくか」
    雄治は少し\ruby{怒}{おこ}った声を出した。

    貴之は\ruby{肩}{かた}をすくめ、腰を上げた。
    「ビール、あるんだろ。
    貰うぜ」

    雄治の返事はないが、冷蔵庫を開けた。
    2ドアタイプの旧式で、二年前に姉の家が買い替えた時、それまで使っていたものを貰ったのだ。
    この前に使っていたのは1ドアだった。
    昭和三十五年に買った\ruby{代物}{しろもの}だ。
    貴之は大学生だった。
    
    ビールの\ruby{中瓶}{なかびん}が\ruby{二本}{にほん}\ruby{冷}{ひ}えていた。
    酒好きの雄治は冷蔵庫からビールを\ruby{絶}{た}やすことがない。
    昔は甘いものになど見向きもしなかった。
    木村屋のあんぱんが\ruby{大好物}{だいこうぶつ}になったのは、六十歳を過ぎてからだ。

    まずはビール瓶を一本取り出し、\ruby{栓}{せん}を抜いた。
    さらに食器棚から勝手にコップを二つ出し、卓袱台に戻った。

    「親父も飲むだろ」

    「いや、今はいらん」
\end{CJK}
\end{document}