\documentclass[chapter3.tex]{subfiles}
\usepackage[overlap, CJK]{ruby}

\begin{document}
\begin{CJK}{UTF8}{ipxm}
    % p.146
    雄治はゆっくりと\ruby{頷}{うなず}いた。
    
    「女の人からの相談だ。
    この手の問題は一番\ruby{苦手}{にがて}だ。」
    
    \ruby{色恋}{いろこい}\ruby{沙汰}{ざた}だな、と\ruby{解}{かい}した。
    雄治は見合い結婚だが、お互い\ruby{婚礼}{こんれい}の当日まで相手のことをよく知らなかったという話だ。
    そんな時代を過ごしてきた人間に恋愛問題を相談する方が非常識だと貴之は思う。

    「適当に書いとけよ」

    「何言ってるんだ。
    そんなわけにいくか」
    雄治は少し\ruby{怒}{おこ}った声を出した。

    貴之は\ruby{肩}{かた}をすくめ、腰を上げた。
    「ビール、あるんだろ。
    貰うぜ」

    雄治の返事はないが、冷蔵庫を開けた。
    2ドアタイプの旧式で、二年前に姉の家が買い替えた時、それまで使っていたものを貰ったのだ。
    この前に使っていたのは1ドアだった。
    昭和三十五年に買った\ruby{代物}{しろもの}だ。
    貴之は大学生だった。
    
    ビールの\ruby{中瓶}{なかびん}が\ruby{二本}{にほん}\ruby{冷}{ひ}えていた。
    酒好きの雄治は冷蔵庫からビールを\ruby{絶}{た}やすことがない。
    昔は甘いものになど見向きもしなかった。
    木村屋のあんぱんが\ruby{大好物}{だいこうぶつ}になったのは、六十歳を過ぎてからだ。

    まずはビール瓶を一本取り出し、\ruby{栓}{せん}を抜いた。
    さらに食器棚から勝手にコップを二つ出し、卓袱台に戻った。

    「親父も飲むだろ」

    「いや、今はいらん」

    % p.147
    「そうなのか。珍しいな」

    「回答を書き終えるまでは酒は飲まん。
    いつもそう言ってるだろ」

    ふうん、と\ruby{頷}{うなず}きながら貴之は自分のコップにビールを\ruby{注}{そそ}いだ。
    
    考え込んでいた雄治が、ゆっくりと貴之の方に顔を\ruby{巡}{めぐ}らせた。

    「父親には\ruby{女房}{にょうぼう}と子供がいるらしい」
    いきなり、そういった。

    はあ、と貴之は口を開けた。
    「何の話だ」

    雄治は、そばに置いてある封筒を摘んだ。

    「相談者だ。
    女性なんだが、父親には\ruby{妻子}{さいし}がいるんだ」

    やはり意味がわからない。
    貴之はビールを\ruby{一口}{ひとくち}飲んでから、コップを置いた。

    「そりゃそうだろう。
    俺の父親にだって、妻と子供がいた。
    妻は死んだけど、子供は生きている。
    この俺だ」

    雄治は顔をしかめ、\ruby{苛立}{いらだ}ったように首を振った。

    「わしの話なんかはしてない。
    そういう意味じゃない。
    父親ってのは、相談者の父親ではなく、子供の父親だ」

    「子供?
    誰の?」

    だから、と雄治はもどかしそうに手を振った。
    「お腹の子供だ。
    相談者の」

    えっ、といってから、ああ、と納得した。

    「そういうことか。
    相談者は妊娠してるわけだ。
    で、相手の男が妻子持ちなんだな」

    % p.148
    「そうだ。
    さっきからそういってるだろう」

    「言い方が悪いんだよ。
    父親って言われたら、誰だって相談者の父親だと思うだろ」

    「それは\ruby{早合点}{はやがてん}というものだ」

    「そうかな」
    貴之は首を\ruby{捻}{ねじ}り、コップに手を伸ばした。
    
    「で、どう思う?」
    雄治が訊いてきた。

    「何が」

    「何を聞いてるんだ。
    相手の男には\ruby{女房}{にょうぼう}と子供がいる。
    そんな男の子供を\ruby{孕}{はら}んだわけだ。
    どうすりゃいいと思う?」


    ようやく相談内容が見えてきた。
    貴之はビールを飲み、ふうっと息を吐いた。

    「全く\ruby{近頃}{ちかごろ}の若い女は\ruby{節操}{せっそう}がないな。
    おまけに馬鹿だ。
    女房持ちと関わって、良いことなんかあるわけない。
    何を考えてるんだ」

    雄治は\ruby{渋面}{じゅうめん}を作り、\ruby{卓袱台}{ちゃぶだい}を\ruby{叩}{たた}いた。

    「\ruby{講釈}{こうしゃく}はいいから、どうすればいいかを答えろ」

    「そんなことは決まってるんだろ。
    \ruby{堕}{お}ろすしかない。
    他にどういう答えがあるんだ」

    雄治はふんと鼻を鳴らし、耳の後ろを\ruby{掻}{か}いた。
    「お前に\ruby{訊}{き}いたのが間違いだった」

    「何だよ、どういう意味だ」

    すると雄治はげんなりしたように口元を\ruby{曲}{ま}げ、相談者の封筒を手でぽんぽんと叩いた。

    「\ruby{堕}{お}ろすしかない、他にどういう答えがあるんだ
    ---
    お前でさえ、そんなふうにいうんだ。
    この相談者だって、まずはそう考えただろうさ。
    その上で悩んでるってことがわからんのか」

    % p.149
    鋭い指摘に貴之は黙り込んだ。
    確かにその通りだ。

    いいか、と雄治はさらにいった。

    「\ruby{堕}{お}ろした方がいいということはわかっているとこの人は書いている。
    相手の男が責任を取ってくれるとは思えないし、
    \ruby{女手}{おんなで}ひとつで育てるとなれば、
    この先、相当苦労するだろうと冷静に\ruby{見極}{みき}めている。
    その上で、どうしても産みたいという気持ちを捨てきれない、
    \ruby{堕}{お}ろすことなど考えられないといっているんだ。
    どうしてだか、わかるか?」

    「さあね。俺にはわからんよ。
    親父にはわかるのか」

    「手紙を読んだからな。
    この人によれば、これは最後のチャンスらしい」

    「最後って?」

    「この機会を\ruby{逃}{のが}せば、自分が子供を産むことはないだろうということだ。
    この人は前に一度結婚していて、どうしても子供ができないんで病院で\ruby{診}{み}てもらったら、
    子供の\ruby{出来}{でき}にくい\ruby{体質}{たいしつ}だとわかったそうなんだ。
    医者からは、子供は\ruby{諦}{あきら}めた方がいいとまで言われたらしい。
    それが理由で結婚生活もうまくいかなかったみたいだ」

    「\ruby{不妊症}{ふにんしょう}ってやつか」

    「とにかくそういう事情だから、この人にとっては最後のチャンスってことになるわけだ。
    ここまで聞けばいくらお前でも、堕ろすしかない、なんて簡単には答えられないとわかるだろう」

    % p.150
    貴之はコップのビールを\ruby{飲}{の}み\ruby{干}{ほ}し、\ruby{瓶}{びん}に手を伸ばした。
    
    「言ってることはわかるけどさあ、やっぱり産むのはやめた方がいいんじゃないか。
    子供がかわいそうだろ。
    きっと、苦労するぜ」

    「だからそれは覚悟していると書いてある」

    「そうは言ってもなあ」
    貴之はコップにビールを\ruby{注}{そそ}いだ後、顔を上げた。
    「だけど、それ、相談って感じじゃないな。
    そこまでいうなら、もう産む気だぜ。
    親父がどう回答しようが、関係ないんじゃないか」

    雄治が頷いた。
    「かもしれん」

    「かもしれんって \ldots\ldots」

    「長年悩みの相談を読んでいるうちに分かったことがある。
    多くの場合、相談者は答えを決めている。
    相談するのは、それが正しいってことを確認したいからだ。
    だから相談者の中には、回答を読んでから、もう一度手紙を\ruby{寄越}{よこ}す者もいる。
    多分回答内容が、自分が思っていたものと違っているからだろう」

    貴之はビールを飲み、顔を\ruby{歪}{ゆが}めた。
    「よくそんな面倒臭いことに何年も付き合ってるな」

    「これも\ruby{人助}{ひとだすけ}けだ。
    面倒臭いからこそ、やり\ruby{甲斐}{がい}がある」

    「全く\ruby{物好}{ものず}きだな。
    だけどそういうことなら、考える必要はないだろ。
    その人は産む気みたいなんだから、頑張って元気な赤ちゃんを産んでください、
    とでも書けばいいじゃないか」


\end{CJK}
\end{document}