\documentclass[chapter3.tex]{subfiles}
\usepackage[overlap, CJK]{ruby}

\begin{document}
\begin{CJK}{UTF8}{ipxm}
    \section*{1}

    % p.137
    \ruby{改札口}{かいさつぐち}を出て\ruby{腕時計}{うでどけい}を見ると、\ruby{二本}{にほん}の\ruby{針}{はり}は午後8時半を少し過ぎたところを指していた。
    おかしいなと思い、\ruby{周囲}{しゅうい}を見回した。
    \ruby{案}{あん}の\ruby{定}{じょう}、\ruby{時刻表}{じこくひょう}の上に取り付けられた時計は、八時四十五分を示している。
    浪矢\ruby{貴之}{たかゆき}は口元を\ruby{歪}{ゆが}め、舌打ちした。
    オンボロ時計め、また\ruby{狂}{くる}ってやがる。

    大学の\ruby{合格}{ごうかく}祝いで父親かもらった時計は、最近になって\ruby{不意}{ふい}に止まることが多くなった。
    20年も使っていれば当然か。
    そろそろクォーツに買い替えようかなと考えた。
    \ruby{水晶}{すいしょう}\ruby{発振}{はっしん}方式の\ruby{画期}{かっき}的な時計は、かつては軽自動車\ruby{並}{な}みの値段がしたが、最近では急速に低\ruby{価格}{かかく}化している。

    駅を出て、商店街を歩いた。
    この時間になっても、まだ開いている店があることに驚いた。
    外から\ruby{覗}{のぞ}いた限りでは、どの店もなかなかに\ruby{繁盛}{はんじょう}しているらしい。
    ニュータウンができて新しい\ruby{住人}{じゅうにん}が増え、駅前商店街の\ruby{需要}{じゅよう}が高まった、と聞いたことがある。

    % p.138
    こんな地方の、ぱっとしない街がねえ、と貴之は意外に思うが、生まれ育った\ruby{土地}{とち}に\ruby{活気}{かっき}が戻っているという話を聞いて悪い気はしない。
    それどころか、せめてうちの店もこの商店街の中にあったならな、などと考えてしまう。

    商店街の\ruby{並}{なら}ぶ通りから\ruby{脇道}{わきみち}に入り、しばらくまっすぐ歩いた。
    すぐに住宅の\ruby{建}{た}ち\ruby{並}{なら}ぶエリアに入った。
    この辺りは来るたびに\ruby{景色}{けしき}が少しずつ変わる。
    新しい家が\ruby{次々}{つぎつぎ}と建っていくからだ。
    それらの\ruby{住人}{じゅうにん}の中には、ここから東京まで通勤している者も珍しくないという。
    特急電車を使っても、二時間はかかるだろう。
    自分にはとてもできない、と貴之は思った。
    彼の現在の住まいは都内の\ruby{賃貸}{ちんたい}マンションだ。
    狭いながらも2LDKで、妻と十歳の息子と三人で暮らしている。

    しかし、と思い直した。
    ここから通うのは無理だが、\ruby{立地}{りっち}\ruby{条件}{じょうけん}について、ある程度は\ruby{妥協}{だきょう}する必要はあるかもしれない。
    人生は、自分の思う通りにならないことの方が多い。
    通勤時間が延びるぐらいのことは我慢すべきだろう。

    住宅地を抜けると、T\ruby{字}{じ}\ruby{路}{ろ}に出た。
    \ruby{右折}{うせつ}し、さらに歩いていく。
    \ruby{緩}{ゆる}やかな上り坂だ。
    この辺りなら、目を\ruby{瞑}{つむ}っていても歩ける。
    どれだけ歩けば、道がどの程度に\ruby{曲}{ま}がっていくか、体が覚えている。
    何しろ、高校を卒業するまで通った道だ。

    やがて右前方に小さな建物が見えてきた。
    \ruby{街灯}{がいとう}は\ruby{点}{とも}っているが、\ruby{看板}{かんばん}の字は\ruby{煤}{すす}けていて読みにくい。
    シャッターは\ruby{閉}{し}まっていた。

    % p.139
    店の前で足を止め、改めて看板を見上げた。
    ナミヤ雑貨店---\ruby{近}{ちか}づけば\ruby{辛}{かろ}うじて読める。

    隣の\ruby{倉庫}{そうこ}との間に、幅一メートルほどの\ruby{通路}{つうろ}がある。
    貴之は、そこから店の裏側に回った。
    小学生の頃は、ここに自転車を止めていた。

    店の裏には\ruby{勝手口}{かってぐち}があった。
    ドアのすぐ横に牛乳箱が取り付けられている。
    牛乳を配達してもらっていたのは、十年ほど前までだ。
    母親が亡くなって、しばらくしてからやめた。
    しかし牛乳箱はそのままだ。

    牛乳箱の\ruby{脇}{わき}にはボタンが付いている。
    押せば、昔はブザーが\ruby{鳴}{な}った。
    今は\ruby{鳴}{な}らない。

    貴之はドアノブを引いた。
    やはり抵抗なく\ruby{開}{あ}いた。
    いつもこうだ。

    \ruby{靴}{くつ}\ruby{脱}{ぬ}ぎには、見慣れたサンダルと、\ruby{古}{ふる}びた\ruby{革靴}{かわぐつ}が並んでいた。
    どちらも\ruby{所有者}{しょゆうしゃ}は同じだ。

    今晩は、と低く声をかけた。
    返事はなかったが、\ruby{構}{かま}わずに進んだ。
    靴を脱ぎ、上がり込んだ。
    入ってすぐのところが\ruby{台所}{だいどころ}だ。
    その先には和室があり、さらにその向こうが\ruby{店舗}{てんぽ}になっている。

    雄治は和室で\ruby{卓袱台}{ちゃぶだい}に向かっていた。
    \ruby{股引}{ももひき}にセーターという\ruby{出}{い}で\ruby{立}{だ}ちで、\ruby{正座}{せいざ}をしている。
    そのまま顔だけをゆっくりと貴之の方に向けた。
    \ruby{老眼鏡}{ろうがんきょう}を鼻先にずらしている。

    「何だ、おまえか」

    「何だ、じゃないよ。
    \ruby{鍵}{かぎ}がかかってなかったぞ。
    \ruby{戸締}{とじ}りはきちんとしろって、いつもいっているだろ」

    % p.140
    「大丈夫だ。
    誰か来たら、すぐにわかる」

    「わからなかったじゃないか。
    俺の声、聞こえなかったんだろ」

    「何か聞こえてたが、考え事をしてたので、返事をするのが面倒だったんだ。」

    「また、そういう\ruby{負}{ま}け\ruby{惜}{お}しみを」
    貴之は\ruby{持参}{じさん}してきた小さな\ruby{紙袋}{かみぶくろ}を\ruby{卓袱台}{ちゃぶだい}に置き、\ruby{胡座}{あぐら}をかいた。
    「ほら、親父の好きな木村屋のあんぱんだ」

    おう、と雄治は目を\ruby{輝}{かがや}かせた。
    「いつもすまんな」

    「別にいいよ、これぐらい」

    雄治は、どっこいしょと立ち上がり、紙袋をつまみ上げた。
    すぐそばの仏壇は扉が開いたままだ。
    そこの台にあんぱんの入った袋を置くと、立ったままで\ruby{鈴}{すず}を二度鳴らし、元の場所に座った。
    \ruby{小柄}{こがら}で\ruby{痩}{や}せているが、八十歳近くになっても姿勢だけは良い。

    「お前、\ruby{晩飯}{ばんめし}は食ったのか」

    「会社の帰りに\ruby{蕎麦}{そば}を食った。
    今夜はこっちに泊まるから」

    「ふうん。\ruby{芙美子}{ふみこ}さんにはいってあるのか」

    「ああ。あいつも親父のことを心配してたぜ。
    体調はどうなんだ」

    「お陰様で問題ない。
    わざわざ様子を見にきてもらうまでもない」

    「せっかく来てやったのに、その言い方はないだろ」

    「心配無用と言ってるだけだ。
    ああそうだ、さっき\ruby{風呂}{ふろ}に入って、\ruby{湯}{ゆ}はそのままにしてある。
    まだ\ruby{冷}{さ}めてないだろうから、好きな時に入ればいい」

    % p.141
    会話の\ruby{間中}{まなか}、雄治の視線は卓袱台の上に向けられていた。
    そこには\ruby{便箋}{びんせん}が広げられている。
    \ruby{傍}{かたわ}らに封筒が置いてあった。
    \ruby{表書}{おもてが}きは、ナミヤ雑貨店様へ、となっている。

    「それ、今夜来たのか」
    貴之は訊いた。

    「いや、届いたのは昨日の深夜だ。
    朝になって、気づいた」

    「それなら、今朝、回答しなきゃいけなかったんじゃないのか」

    『ナミヤ雑貨店』への悩み相談の回答は、\ruby{翌朝}{よくあさ}牛乳箱に入れられる --- それが雄治の作ったルールのはずだ。
    そのため雄治は午前五時半には起きる。

    「いや、\ruby{夜中}{やちゅう}だということで相談者も気を\ruby{遣}{つか}ったらしい。
    回答は一日遅れでいいと書いてある」

    「ふうん、そうなのか」

    おかしな話だ、と貴之は思った。
    なぜ雑貨屋の\ruby{店主}{てんしゅ}が、他人の悩み相談に応じねばならないのか。
    もちろん、こうなってしまった経緯はわかっている。
    何しろ、週刊誌が\ruby{取材}{しゅざい}に来たほどなのだ。
    あの直後は相談件数が増えた。
    真面目な内容もあったが、多くがふざけたものだった。
    明らかに\ruby{嫌}{いや}がらせと思われるものも少なくなかった。
    極めつけは一晩で三十通以上の悩みが持ち込まれたことだ。
    明らかに一人の手によるものだった。
    内容は全てでたらめなものだった。
    ところが雄治は、それらにさえも回答をしようとした。
    さすがにその時には、「やめろよ、そんなこと」と貴之は雄治にいった。

\end{CJK}
\end{document}