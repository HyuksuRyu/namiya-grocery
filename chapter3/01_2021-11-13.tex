\documentclass[chapter3.tex]{subfiles}
\usepackage[overlap, CJK]{ruby}

\begin{document}
\begin{CJK}{UTF8}{ipxm}
    \section*{1}

    \ruby{改札口}{かいさつぐち}を出て\ruby{腕時計}{うでどけい}を見ると、\ruby{二本}{にほん}の\ruby{針}{はり}は午後8時半を少し過ぎたところを指していた。
    おかしいなと思い、\ruby{周囲}{しゅうい}を見回した。
    \ruby{案}{あん}の\ruby{定}{じょう}、\ruby{時刻表}{じこくひょう}の上に取り付けられた時計は、八時四十五分を示している。
    浪矢\ruby{貴之}{たかゆき}は口元を\ruby{歪}{ゆが}め、舌打ちした。
    オンボロ時計め、また\ruby{狂}{くる}ってやがる。

    大学の\ruby{合格}{ごうかく}祝いで父親かもらった時計は、最近になって\ruby{不意}{ふい}に止まることが多くなった。
    20年も使っていれば当然か。
    そろそろクォーツに買い替えようかなと考えた。
    \ruby{水晶}{すいしょう}\ruby{発振}{はっしん}方式の\ruby{画期}{かっき}的な時計は、かつては軽自動車\ruby{並}{な}みの値段がしたが、最近では急速に低\ruby{価格}{かかく}化している。

    駅を出て、商店街を歩いた。
    この時間になっても、まだ開いている店があることに驚いた。
    外から\ruby{覗}{のぞ}いた限りでは、どの店もなかなかに\ruby{繁盛}{はんじょう}しているらしい。
    ニュータウンができて新しい\ruby{住人}{じゅうにん}が増え、駅前商店街の\ruby{需要}{じゅよう}が高まった、と聞いたことがある。



\end{CJK}
\end{document}