\documentclass[chapter3.tex]{subfiles}
\usepackage[overlap, CJK]{ruby}

\begin{document}
\begin{CJK}{UTF8}{ipxm}
    % p.151
    すると雄治は息子の顔を見て口をへの\ruby{字}{じ}にし、ゆらゆらと頭を振った。
    
    「やっぱりお前は何も分かってない。
    確かに手紙からは産みたいという気持ちがヒシヒシと伝わってくる。
    しかし大事なのは、気持ちと用意は別だってことだ。
    もしかしたらこの人は、産みたいと強く思いつつも、
    堕ろすしかないと頭では分かっていて、
    その決心を固めたくて手紙を書いたのかもしれない。
    だとしたら、産みなさいなんて書いたら、
    全くの逆効果だ。
    余計に苦しめることになる」

    貴之は指先でこめかみを押した。
    頭が痛くなってきた。

    「俺なら、勝手にしろと書くな」

    「心配せんでも、誰もお前には回答を求めとらん。
    とにかくこの文面から、相談者の心理を読まなきゃならんのだ」
    雄治は再び腕組みをした。

    大変だな、と貴之は\ruby{他人事}{ひとごと}ながら思う。
    だがこうして回答を考えるのが、雄治にとっては何よりも楽しいのだろう。
    それだけに用件を切り出しにくかった。
    貴之が今夜ここへ来たのは、単に\ruby{老}{お}いた父親の様子を見るためだけではないのだ。

    「親父、ちょっといいかな。
    俺からも話があるんだけど」

    「何だ。
    見ればわかると思うが、今、忙しんだ」

    「そんなに時間は取らせないよ。
    それに、忙しいと言ったって、ただ考え込んでるだけじゃないか。
    少し違うことを考えた方が、良い案が浮かぶかもしれないぜ」

    %p.152
    それもそうだとでも思ったのか、雄治が\ruby{仏頂面}{ぶっちょうずら}を息子に向けた。
    「一体、何だ」

    貴之は\ruby{背筋}{せすじ}を伸ばした。

    「\ruby{姉貴}{あねき}から聞いた。店の方、かなり悪いみたいだな」

    途端に雄治は顔を\ruby{顰}{しか}めた。
    「\ruby{頼子}{よりこ}のやつ、余計なことを」

    「心配して知らせてくれたんだ。
    娘なんだから当然だろ」

    頼子は昔、\ruby{税理士}{せいりし}\ruby{事務所}{じむしょ}に勤めていた。
    その時の経験を\ruby{生}{い}かし、『ナミヤ雑貨店』の確定\ruby{申告}{しんこく}を、全て彼女が処理しているのだ。
    ところが先日、今年の分を済ませた彼女が、貴之に電話をかけてきた。

    「ひどいわよ、うちの店。
    赤字なんてもんじゃない。
    \ruby{真っ赤}{まっか}っか。
    あれじゃあ誰が確定申告しても同じよ。
    \ruby{節税}{せつぜい}対策なんて必要ない。
    正直に申告しても、税金なんて一銭も払わなくていいもの」

    そんなにひどいのかと貴之が訊くと、
    「お父さん本人が申告に行ってたら、生活\ruby{保護}{ほご}の申請を勧められてたかも」
    という答えが返ってきたのだった。

    貴之は父親の方に向き直った。

    「なあ、そろそろ店を\ruby{畳}{たた}んだ方がどうだ。
    この\ruby{辺}{へん}の客は、今では駅前の商店街に行くだろ?
    あの駅ができる前は、バスの\ruby{停留所}{ていりゅうじょ}が近いってことで、この\ruby{辺}{あた}りでも商売ができたけど、もう無理だ。
    諦めた方がいい」

    %p.153
    雄治はゲンナリしたような顔で\ruby{顎}{あご}を\ruby{擦}{こす}った。

    「店を畳んで、どうしろっていうんだ」

     


\end{CJK}
\end{document}